\documentclass[../report.tex]{subfiles}
\begin{document}
    
    \begin{frame}[t]
        \frametitle{6a. Discussion}
        \begin{normalsize}
            \begin{itemize}
                \setlength\itemsep{1em}\fontsize{6pt}{6pt}
                
                \item[]{\textbf{Between all color channels, which channel, in your opinion, most resembles a gray-scale conversion of the original.  Why do you think this?  Does it matter for each respective image? (For this problem, you will have to read a bit on how the eye works/cameras to discover which channel is more prevalent and widely used)} }
                
                \item[]{\selectfont\textcolor{blue}{I think \\
my answer \\
is ... \\
}}
                
            \end{itemize}
        \end{normalsize}
    \end{frame}

    \begin{frame}[t]
        \frametitle{6b. Discussion}
        \begin{normalsize}
            \begin{itemize}
                \setlength\itemsep{1em}\fontsize{6pt}{6pt}
                
                \item[]\textbf{What does it mean when an image has negative pixel values stored?  Why is it important to maintain negative pixel values?}
                
                \item[]{\selectfont\textcolor{blue}{Pixel values are somewhat arbitrary.\\
A negative value could just be a low value.\\
What is important is the range of acceptable \\
pixel values, for example, you could have an\\
image where -10 is black and 10 is white.\\
Or you could have an image where 0 is black \\
and 20 is white.  It's important to keep \\
negative pixel values because if we force \\
them to be 0, we may lose information.
}}
                
            \end{itemize}
        \end{normalsize}
    \end{frame}

    \begin{frame}[t]
        \frametitle{6c. Discussion}
        \begin{normalsize}
            \begin{itemize}
                \setlength\itemsep{1em}\fontsize{6pt}{6pt}
                
                \item[]\textbf{In question 5, noise was added to the green channel and also to the blue channel. Which looks better to you? Why? What sigma was used to detect any discernible difference?}
                
                \item[]{\selectfont\textcolor{blue}{I think \\
my answer \\
is ... \\
}}
                
            \end{itemize}
        \end{normalsize}
    \end{frame}
    
\end{document}